\documentclass{article}
\usepackage{amsmath}
\usepackage{xeCJK}
\setCJKmainfont{Hiragino Mincho ProN} % Macの日本語フォント
\title{LaTeXの基礎}
\author{山田太郎}
\date{\today}

\begin{document}
\maketitle

ここに本文を書く。これは\textbf{太字のテキスト}です。

これは\underline{下線付きのテキスト}です。

これはインライン数式の例です。$E=mc^2$はエネルギーと質量の関係式です。

\[
  E=mc^2
\]

\begin{equation}
  E=mc^2
\end{equation}

\begin{align}
  a & =b+c   \\
    & =d+e+f \\
    & =g+h
\end{align}

\begin{equation}
  f(x) =
  \begin{cases}
    x^2 & (x \geq 0) \\
    -x  & (x < 0)
  \end{cases}
\end{equation}

\[
  \alpha, \beta, \gamma, \Gamma, \Delta, \Theta
\]

\[
  \frac{a+b}{c+d}
\]

\[
  x^2,\quad x^{2+3}
\]

\[
  \int_{0}^{1}x^2\,dx
\]

\[
  \begin{bmatrix}
    1 & 2 & 3 \\
    4 & 5 & 6 \\
    7 & 8 & 9
  \end{bmatrix}
\]

\begin{align}
  x+y  & =5 \\
  2x-y & =3
\end{align}

\[
  f(x)=
  \begin{cases}
    x^2 & (x\geq 0) \\
    -x  & (x<0)
  \end{cases}
\]

\[
  \frac{dy}{dx}=x^2+2x+1
\]

\[
  \int_{a}^{b}x^2\,dx
\]

\[
  \mathcal{L}\{f(t)\}=\int_{0}^{\infty}e^{-st} f(t)\,dt
\]

\[
  \begin{bmatrix}
    a & b \\
    c & d
  \end{bmatrix}
  \cdot
  \begin{bmatrix}
    x \\
    y
  \end{bmatrix}
  =
  \begin{bmatrix}
    ax+by \\
    cx+dy
  \end{bmatrix}
\]

\[
  f(x)=
  \begin{cases}
    x^2  & \text{if } x>0, \\
    0    & \text{if } x=0, \\
    -x^2 & \text{if } x<0
  \end{cases}
\]

\[
  g(x)=\left\{
  \begin{array}{ll}
    x^3+3x+1,               & \text{if } x > 1                           \\
    \dfrac{x^2 - 1}{x + 1}, & \text{if } x \leq 1 \text{ and } x \neq -1
  \end{array}
  \right.
\]



\end{document}
