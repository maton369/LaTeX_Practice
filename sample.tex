\documentclass{article}
\usepackage{amsmath}
\usepackage{xeCJK}
\setCJKmainfont{Hiragino Mincho ProN} % Macの日本語フォント
\title{LaTeXの基礎}
\author{山田太郎}
\date{\today}

\begin{document}
\maketitle

ここに本文を書く。これは\textbf{太字のテキスト}です。

これは\underline{下線付きのテキスト}です。

これはインライン数式の例です。$E=mc^2$はエネルギーと質量の関係式です。

\[
  E=mc^2
\]

\begin{equation}
  E=mc^2
\end{equation}

\begin{align}
  a & =b+c   \\
    & =d+e+f \\
    & =g+h
\end{align}

\begin{equation}
  f(x) =
  \begin{cases}
    x^2 & (x \geq 0) \\
    -x  & (x < 0)
  \end{cases}
\end{equation}

\[
  \alpha, \beta, \gamma, \Gamma, \Delta, \Theta
\]

\[
  \frac{a+b}{c+d}
\]

\[
  x^2,\quad x^{2+3}
\]

\end{document}
